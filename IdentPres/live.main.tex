%! TeX program = luatex
\documentclass[9pt]{beamer}

\usepackage{physics, amsmath, amsfonts,amssymb}

\usetheme{metropolis}
\title[]{When does a system admit a equilirbium probability distribution}
\author[Sabarno Saha]{Sabarno Saha \inst{1} }
\institute[IISERK]{\inst{1} Department of Physical Sciences IISERK}
\date{\today}
  \metroset{block=fill}

\begin{document}

\documentclass[9pt]{beamer}

\usepackage{physics, amsmath, amsfonts,amssymb}

\usetheme{metropolis}
\title[]{When does a system admit a equilirbium probability distribution}
\author[Sabarno Saha]{Sabarno Saha \inst{1} }
\institute[IISERK]{\inst{1} Department of Physical Sciences IISERK}
\date{\today}
  \metroset{block=fill}

\begin{document}

\frame{\titlepage}

\begin{frame}
  \frametitle{Introduction}
  \begin{enumerate}
    \item Stochastic Description
      \begin{itemize}
        \item Stochastic Processes
        \item Markov Chains
        \item CTMC 
        \item Transition Rate Matrix
      \end{itemize}
    \item Steady State and Equilibrium Distribution
      \begin{itemize}
        \item Conditions
        \item Detailed Balance Condition
      \end{itemize}
    \item Perrin Frobenius Theorem
      \begin{itemize}
        \item Irreducibility
        \item Strongly connected graphs
        \item Stationary State distribution
      \end{itemize}
  \end{enumerate}
\end{frame}

\begin{frame}
  \frametitle{Introduction(Contd.)}
  \begin{enumerate}\setcounter{enumi}{4}
    \item Graph/ Network Theory
      \begin{itemize}
        \item Graphs
        \item Trees
        \item Handshaking lemma
        \item An "obvious" theorem
      \end{itemize}
    \item Equilibrium Distribution
  \end{enumerate}

\end{frame}


\section{A Stochastic Description}
\begin{frame}
  \frametitle{Stochastic Process}
  \metroset{block=fill}
  \begin{block}{Stochastic Process}

    A stochastic process is a sequence of random variables where the indexing of the variables
    often carries the notion of time.

  \end{block}
  For example, we have Brownian motion, which is represented using the Wiener process
  (a stochastic process).

\end{frame}
\begin{frame}
  \frametitle{Markov Chains}
  Most of the physical processes that we study in classical statistical physics is modelled as
  Markov Chains. 
  \begin{block}{Markov Property}
    Let $ \{X_n\}$ be a stochastic process.
    The markov property is defined as
    $$\mathbb{P}(X_{n+1}|X_n,X_{n-1},...,X_0) = \mathbb{P}(X_{n+1}|X_n)$$
    Any stochastic process satisfying the Markov property is called a Markov Chain.
  \end{block}

\end{frame}

\begin{frame}
  \frametitle{Continuous Time Markov Chains(CTMC)}
  These are markov chains with the index as time $\{X(t)\}_t$. Let us define a state space as the 
  set of all values $S$ a random variable can assume.
\end{frame}

\begin{frame}
  \frametitle{Transition Rate Matrix}
\end{frame}

\section{Steady State \& Equilibrium Distribution}

\begin{frame}
  \frametitle{Conditions}
\end{frame}

\begin{frame}
  \frametitle{Detailed Balance Condition}
\end{frame}

\section{Perrin Frobenius Theorem}

\begin{frame}
  \frametitle{Irreducibility}

\end{frame}

\begin{frame}
  \frametitle{Strongly Connected Graphs}
\end{frame}

\begin{frame}
  \frametitle{Stationary State Distribution}

\end{frame}

\section{Graphs/ Network Theory}

\begin{frame}
  \frametitle{Trees}

\end{frame}
\begin{frame}
  \frametitle{Handshaking Lemma}

\end{frame}
\begin{frame}
  \frametitle{An "obvious" Theorem}

\end{frame}
\section{Equilibrium Distribution}
\begin{frame}
  \frametitle{Acyclic networks}

\end{frame}
\begin{frame}
  \frametitle{Cyclic Networks}

\end{frame}

\begin{frame}[standout]

  Questions?
\end{frame}
\section{Danke Schon}
\end{document}

\end{document}
