\documentclass{beamer}

\usetheme{Madrid}
\title[]{Bayesian Inference and Information based model check of Langevin Systems}
\author[Sabarno Saha]{Sabarno Saha \inst{1} \and Dr. Rajesh Singh \inst{2}}
\institute[IISERK]{\inst{1} Department of Physical Sciences IISERK \and %
                      \inst{2} Department of Physical Sciences IIT Madras}
\date{\today}

\begin{document}

\frame{\titlepage}

\begin{frame}
  \frametitle{Introduction}
  \begin{enumerate}
    \item Stochastic Thermodynamics
      \begin{itemize}
        \item Stochastic Processes
        \item Brownian Motion
        \item Stochastic Differential equation 
        \item Stochastic Integrals
        \item Langevin Equation 
        \item Fokker-Planck Equation
        \item Euler-Maruyuma Integrator
      \end{itemize}
    \item Bayesian Inference
      \begin{itemize}
        \item Bayes Theorem
        \item Prior Assignment
        \item Example
      \end{itemize}
  \end{enumerate}
\end{frame}

\begin{frame}
  \frametitle{Introduction(Contd.)}
  \begin{enumerate}\setcounter{enumi}{3}
    \item Information Theory
      \begin{itemize}
        \item Shannon Information
        \item Fisher Information
        \item Kullback Liebler Divergence
      \end{itemize}
    \item Nested Sampling
      \begin{itemize}
        \item Likelihood Function
        \item MCMC
        \item Evidence Calculation
      \end{itemize}
    \item Model Check
      \begin{itemize}
        \item Information check
        \item Scaling of Steps
        \item p-value check
      \end{itemize}
  \end{enumerate}

\end{frame}

\begin{frame}
  \frametitle{Stochastic Thermodynamics}
 \begin{definition}
  
   A stochastic process is a sequence of random variables where the indexing of the variables
   often carries the notion of time.
  
 \end{definition}
 For example, we have Brownian motion, which is represented using the Wiener process
 (a stochastic process).
  
\end{frame}
\end{document}
