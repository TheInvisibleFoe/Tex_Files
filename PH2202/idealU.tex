    \section{Introduction} 
    The course introduces us to the fundamentals of thermal physics and will end with statistical physics. The instructor for this course is Dr. Koushik Datta. 
    The recommended reading for this course is "Heat and Thermodynamics" by Zemansky and Dittman. 
    \subsection{Readings} 
    \begin{itemize}
        \item \emph{An Introduction to Thermal Physics } by D.V. Schroeder
        \item \emph{Thermodynamics} by Enrico Fermi
    \end{itemize}
    
    
    \section{Internal energy of Ideal Gas}
    We use some elementary equations already taught in CH1201. We will be using the ideal gas 
    equation and the $1^{st}$ Law of Thermodynamics. 
    \begin{align*}
        PV &= NRT \\ 
        dU &= TdS - pdV + \mu dN
    \end{align*}
    We assume the fact that the internal energy of the system, something we will define later,
    depends on the variables entropy(S), volume(V) and no of moles(N). The first term of equation
    2 above change in heat energy $dQ = TdS$ and work done on the system $dW = -PdV$ and 
    chemical potential $\mu dN$. 
    \begin{align}
        dU &= \pdv{U}{S}dS - \pdv{U}{V}dV + \pdv{U}{N}dN \\ 
        % \tag{Comparing this to the first law} 
        T &= \pdv{U}{S}dS\\ 
        P &= - \pdv{U}{V}dV \\ 
        \mu &= \pdv{U}{N}dN 
    \end{align}
    The chemical potential is a new term added in this course. %definition to added here:
    So chemical potential essentially refers to the change in internal energy on adding or 
    subtracting a molecule. Essentially when we add an infinitesimally small number of molecules
    say dN, we have the chemical potential term to be $\mu dN$.%\\[5]
    
    \begin{problem}
      We need to find a closed form expression of the internal energy of an ideal gas in terms 
      of its state variables
    \end{problem}

    \\[5]
    \emph{Answer: } We introduce one more equation into solving for the closed form expression i.e. the 
    equipartition of energy.
    \begin{align}
        U &= \frac{3}{2} NRT \nonumber\\ 
        \Rightarrow T & = \frac{2U}{3NR} 
    \end{align}
    From here its just algebra to find the solution to the problem
    \begin{align}
        \pdv{U}{S} = T = \frac{2U}{3NR}\nonumber\\ 
        % \tag{Integrating the equation gives us.}
        \Rightarrow \ln (U) = \frac{2S}{3NR}+ f(V,N) 
    \end{align}
    So we now need to obtain the function \(f\). For this again we will use a different 
    thermodynamic equality namely the ideal Gas equation.
    \begin{align*}
        & PV = NRT\\
        \Rightarrow & P = \frac{NRT}{V}\\ 
        \tag{Using (3)}
        & \pdv{U}{V} = -P = \frac{NRT}{V} = -\frac{2U}{3V} \\ 
        \tag{Differentiating (6)}
        & \pdv{(\ln U)}{V} = \pdv{f}{V} \\ 
        &\Rightarrow \frac{1}{U}\pdv{U}{V} = \pdv{f}{V}\\ 
        \tag{Substituting the deriavtive}
        &\Rightarrow \frac{1}{U} \qty(-\frac{2U}{3V}) = \pdv{f}{V} \\ 
        & \Rightarrow \partial{f} = -\frac{2}{3} \frac{\partial{V}}{V} \\ 
        \tag{Integrating}
        & \Rightarrow f = -\frac{2}{3} \ln(V) + \ln(g(N))
    \end{align*} 
    Other than the fucntion g(N) we have found the closed form expression of internal energy of 
    ideal gas which is
    \begin{align}
        &\ln U = \frac{2S}{3NR} - \frac{2}{3} \ln(V) + \ln(g(N)) \nonumber\\ 
        \Rightarrow & U = g(N)V^{-2/3}\exp\qty(\frac{2S}{3NR})  
    \end{align}
    Now to figure out the value of g(N), we need to apply a property of the variables we are working
    with. Now given a thermodynamic variable, it can either be intensive or extensive in nature.
    Intensive variables or properties are such properties that do not depend on the amount of substance
    we are measuring the variable of. For example, the melting point of a liquid is an intensive property
    i.e. if we add more liquid, the melting point still remains the same. In contrast there are 
    extensive properties or variable which depend on the amount of the substance taken. For example,
    mass is a very trivial example of an extensive property. \\ 
    So all the properties that we are dealing with here are extensive. I will try to give a motivation 
    of why they are extensive, but the fact that a property is extensive or intensive is a purely 
    experimental result. \\ 

    We first talk about entropy, which is a concept that measures the disorder in a system. Suppose
    you have a system A and then a system B , the disorder of the combined system will depend on the 
    combined A and B, meaning it is not an inherent property of the substance we are taking. Same goes 
    the case for volume. I mean if we add two systems with different volumes, their resultant volume 
    will be the combined volume. The no of moles(N) is also an extensive property, since it is just a 
    scaled analogy to mass. \\ 
    Similary if we talk about the equipartition of energy, then the internal energy is directly dependent
    on the no of moles, which makes it an extensive property. \\ 

    Thus we will scale the system now where 
    \(S\mapsto \lambda S, V \mapsto \lambda V \quad \text{and} \quad N\mapsto \lambda N \) where 
   \(\lambda\) is some scaling factor. As a result \(U\mapsto \lambda U\). Keeping this in mind we 
   perform the aforementioned mapping on equation (7).
   \begin{align*}
       U' &= g(\lambda N)(\lambda V)^{-2/3}\exp\qty(\frac{2 \lambda S}{3 \lambda NR}) \\ 
          & = g(\lambda N) \lambda^{-2/3} V^{-2/3}\exp\qty(\frac{2S}{3NR}) \\ 
          & = g(\lambda N) \lambda^{-2/3} \frac{U}{g(N)} \\ 
       \Rightarrow \lambda U &=  g(\lambda N) \lambda^{-2/3} \\ 
       \Rightarrow g(\lambda n) &= \lambda^{5/3}g(N)
   \end{align*}
   This is a fairly easy function equation to solve we just set N=1 and \(\lambda = N\). 
   \begin{align}
       \tag{where k is some constant}
       g(N) = kN^{5/3}
   \end{align}
   Now plugging this equation in (7), We get our final closed form solution of the problem we want.\\
   \begin{center}
       \large
       \fbox{ 
       \begin{equation*}
           U = kN^{5/3} V^{-2/3}\exp\qty(\frac{2S}{3NR})  
       \end{equation*}
       }
   \end{center}
   {\hfill \blacksquare}

