\documentclass[a4paper]{article}

\usepackage[utf8]{inputenc}
\usepackage[T1]{fontenc}
\usepackage{textcomp}
\usepackage[dutch]{babel}
\usepackage{amsmath, amssymb,physics}
\usepackage{preamble}
\usepackage{transparent}
\newcommand{\incfig}[1]{%
    \def\svgwidth{\columnwidth}
    \import{./figures/}{#1.pdf_tex}
}
\pdfsuppresswarningpagegroup=1
\title{PH2202: Thermal Physics}
\begin{document}
\maketitle
    \newpage
    \tableofcontents
    \newpage 
    \section{Introduction}
    The course introduces us to the fundamentals of thermal physics and will end with statistical
    physics. 
    
    \section{Internal energy of Ideal Gas}
    We use some elementary equations already taught in CH1201. We will be using the ideal gas 
    equation and the $1^{st}$ Law of Thermodynamics. 
    \begin{align*}
        PV &= NRT \\ 
        dU &= TdS - pdV + \mu dN
    \end{align*}
    We assume the fact that the internal energy of the system, something we will define later,
    depends on the variables entropy(S), volume(V) and no of moles(N). The first term of equation
    2 above change in heat energy $dQ = TdS$ and work done on the system $dW = -PdV$ and 
    chemical potential $\mu dN$. 
    \begin{align*}
        dU &= \pdv{U}{S}dS - \pdv{U}{V}dV + \pdv{U}{N}dN \\ 
        \tag{Comparing this to the first law} 
        T &= \pdv{U}{S}dS\\ 
        P &= - \pdv{U}{V}dV \\ 
        \mu &= \pdv{U}{N}dN 
    \end{align*}
    The chemical potential is a new term added in this course. %definition to added here:
    So chemical potential essentially refers to the change in internal energy on adding or 
    subtracting a molecule. Essentially when we add an infinitesimally small number of molecules
    say dN, we have the chemical potential term to be $\mu dN$.\\[5]
    \fbox{Goal: We need to find a closed form expression for the internal energy of an ideal gas}
    \\[5]
    We introduce one more equation into solving for the closed form expression i.e. the 
    equipartition of energy.
    \begin{align*}
        U &= \frac{3}{2} NRT \\ 
        \Rightarrow T & = \frac{2U}{3NR}\\ 
    \end{align*}
    \begin{align*}
        \pdv{U}{S} = T = \frac{2U}{3NR}\\ 
        \tag{Integrating the equation gives us.}
        \Rightarrow \ln (U) = \frac{2S}{3NR}+ f(V,N) \\ 
        \tag{Putting this in the ideal gas equation}
    \end{align*}

\end{document}
