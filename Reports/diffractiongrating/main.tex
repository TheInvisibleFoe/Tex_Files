\documentclass[a4paper]{article}

\usepackage[utf8]{inputenc}
\usepackage{mlmodern}
\usepackage[T1]{fontenc}
\usepackage{textcomp}
\usepackage[dutch]{babel}
\usepackage{amsmath, amssymb}
\usepackage{preamble}
\usepackage{transparent}
\newcommand{\incfig}[1]{%
    \def\svgwidth{\columnwidth}
    \import{./figures/}{#1.pdf_tex}
}
\pdfsuppresswarningpagegroup=1
\title{Diffraction Grating}
\begin{document}
\maketitle
% \paragraph{\entering Aim}
\renewcommand{\abstractname}{Aim}
\begin{abstract}
    In this experiment we first find the distance between slits ina diffraction grating first 
    using a LASER of known wavelength. Then we use that data to calculate the wavelength of 
    another LASER using the previously calculated slit spacing.
\end{abstract}
\section{Theory}
Diffraction is the phenomenon where the wave bends around corners 
while interference is the phenomenon where two or more waves meet and interact. 
The name diffraction grating is a misnomer as the the phenomenon we see here is 
just N-slit interference rather than diffraction. What essentially we have here is a bunch 
of very small slits spaced at a distance \(d\) to find  which is the aim of our experiment.

Now we come to the phenomenon of interference. There are some important conditions that have 
to be met for interference to occur, a concept called Coherence. Physically, it represents the 
ability of a wave two waves to interfere. Coherence dictates visibility in the interference 
pattern. Interference requires the interfering waves to be coherent, something that can be 
achieved by having a constant phase relative phase difference between the interfering waves. 
Now keeping a broader discussion of coherence, we proceed to the nice mathematics that lies
behind interference. We will take an approach that is purely geometric in nature(Thanks Morin)
\subsection{N-slit interference}
First we have to address how to write a wave mathematically. A very general way is to use 
complex exponentials, which takes care of the fact that it includes both the sine and the cosine
components of a wave. 

\end{document}
