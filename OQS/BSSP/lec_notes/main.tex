\documentclass{scrartcl}

\usepackage[linear]{handout}
\usepackage{bbm}
% \usepackage{gfsartemisia}
\ihead{\sffamily\bfseries\footnotesize{Lecture Notes}}
\ohead{\sffamily\footnotesize\textbf{BSSP 2021}}

\title{
        \Large\textsc{BSSP 2021} \\
        \vspace{10pt}
        % \Large\textsc{Experiment 02} \\
        % \vspace{0.1cm}
        \Huge \textbf{Introduction to Open Quantum Systems} \\
}

% \subtitle{}

\author{Sabarno Saha \\ \texttt{22MS037}}

\date{\normalsize
        \textit{Indian Institute of Science Education and Research, Kolkata, \\
        Mohanpur, West Bengal, 741246, India.}
        % \vspace{10pt}
        % \today
}
\newcommand{\1}{\mathbbm{1}}
\newcommand{\ichi}{\tilde{\chi}}
\newcommand{\irho}{\tilde{\rho}}
\newcommand{\ihsr}{\tilde{H}_{SR}}

\graphicspath{{./images/}}

\begin{document}
\maketitle
\tableofcontents
\pagebreak{}
\section*{Course Outline}
This is a collection of my notes on a series of five lectures delivered by Dr Manas Kulkarni for the Bangalore School on Statistical Physics-XII 2021.
\begin{itemize}
    \item Lecture 1
    \begin{enumerate}
        \item Motivation and General Construction of Open Quantum Systems
        \item System Reservoir Approach
    \end{enumerate}
    \item Lecture 2
    \begin{enumerate}
        \item Quantum Master Equation(QME) : A general Setup
        \item Application to damped harmonic oscillator
    \end{enumerate}
    \item Lecture 3
    \begin{enumerate}
        \item Quantum Langevin Equation(QLE)
        \item Comparisons between QME and QLE
        \item Transport through a system coupled to multiple reservoirs
    \end{enumerate}
    \item Lecture 4
    \begin{enumerate}
        \item Open two-level/Multi-level systems: Perturbative Approach and exact results
        \item Jaynes-Cummings Model: Exact Solutions
    \end{enumerate}
    \item Lecture 5
    \begin{enumerate}
        \item Dicke Model and Quantum Phase transitions
        \item Spectral Signatures in closed and open Dicke Model
    \item Connections to Hermitian and Non-Hermitian Random Matrix Theory.
    \end{enumerate}
\end{itemize}
There are also 3 tutorials accompanying the 5 lectures 
\begin{itemize}
    \item Tutorial 1
    \begin{enumerate}
        \item Density matrix Approach to quantum Mechanics
        \item Quantum Mechanics of Composite Systems
        \item Partial Traces and Reduced Density Matrices of Subsystems
    \end{enumerate}
    \item Tutorial 2
    \begin{enumerate}
        \item Numerical algorithm for the solution of the Driven Dissipative Jaynes-Cummings model.
        \item Numerical implementation using MATLAB.
    \end{enumerate}
    \item Tutorial 3 (July 9, Friday, 2021)
    \begin{enumerate}
        \item Numerical algorithm for finding the spectrum of the Liouvillian of the dissipative Dicke model.
        \item Numerical implementation using MATLAB.
    \end{enumerate}

\end{itemize}
\pagebreak{}
\section{Introduction}
We want to deal with problems where we have a quantum system interacting with some "Reservoir/ Bath/ 
Environment". Most of the systems that we have dealt with so far were closed systems which were isolated to 
the outer environment. We now relax the isolation condition and open up the system to the environment. 
This gives rise to Open Quantum Systems.

The subject aims to introduce the concept of dissipation and drive into quantum mechanics. If these 
elements are naively phenomenologically modeled and massaged into quantum mechanics, it might introduce 
inconsistencies in Heisenberg's Uncertainty Principle or the Commutation relations.

A helpful reference is found at
\subsection{Examples of Such Systems}
\subsubsection{Damped Quantum Harmonic Oscillator}
Most formal treatments of Quantum Mechanics use the quantum harmonic oscillator as the first toy model solved analytically. The damped harmonic oscillator is a toy example used here. This will be solved properly in Lecture 2. 

This can be viewed as a single mode of an electromagnetic field in a lossy cavity (leaky cavity, cavity with imperfect mirrors). A nice solution can also be found at
\subsubsection{Damped two level systems}
These are two-level systems such as qubits that are subject to decay or dephasing due to its coupling to the environment
\section{The system Reservoir approach}
The aim here is to model the interaction between the system "S" and the environment "R" using a general Hamiltonian of the form
\begin{align}
    H  &= H_S \otimes \I_R + \I_S \otimes H_R + H_{SR}(t) \\ \nonumber
    & = H_S + H_R + H_{SR}
\end{align}

where $H_S$ and $H_R$ is the Hamiltonian governing the evolution of the system and the reservoir respectively and $H_{SR}$ is the Hamiltonian governing the coupling between the system and the reservoir. In most cases, the evolution of the reservoir doesn't concern us and it can be specified using its properties like its temperature, energy, or the density of states. 
Let us assume that the reservoir is large compared to the system in question. This means that the number of 
degrees of freedom assosciated with the reservoir is much larger than the number of degrees of freedom associated with the system.
The evolution of the system is out main concern now, rather than the whole system+reservoir $S \otimes R$. Let $\chi (t)$ be the density operator for the whole system $S\otimes R$. Thus the reduced density matrix for the system is given by taking a partial trace over the states in the Hilbert space of the reservoir. The Density Matrix approach to Quantum Mechanics is discussed later under the Section "Tutorial 2". The reduced density matrix is given by 
\begin{equation}
    \rho (t) = Tr_R[\chi (t)]
\end{equation}
where $\rho$ is the reduced density matrix over the system of interest, $Tr_R[]$ is the partial trace over the reservoir states and $\chi$ is the density operator for the system $S\otimes R$.
\subsection{Liouville von-Neuman Equation}
Let $\hat{O}$ be an operator acting on the Hilbert Space associated with the System S. Then the average of 
the operator can be computed its average in the Schroedinger Picture using the already known density matrix 
$\rho(t)$. Note that since $\hat{O}$ acts only on the Hilbert Space of S, $\hat{O} = \hat{O}_S \otimes 
\mathbb{I}_R$

\begin{align}
    \expval{\hat{O}} &= Tr_{S \otimes R}[\hat{O} \chi (t)] = Tr_S[\hat{O}_S Tr_R[\chi (t)]] \\ \nonumber
    &= Tr_S[\hat{O} \rho (t)] 
\end{align}
The goal is to obtain the evolution of the density matrix $\rho(t)$ with time. This is given by the Liouville von-Neumann equation
\begin{equation}
    \frac{d\chi (t)}{dt} = \dot{\chi}(t)= \frac{1}{i \hbar}[H, \chi (t)] 
\end{equation}
where $H$ is the full Hamiltonian $H = H_S + H_R + H_{SR}$.  

The problem is taken to the interaction picture where the operators evolve with time 
using the time evolution operator $U_0$ corresponding to uncoupled Hamiltonian $H_0 = H_S + H_R$. This seperates the
rapid motion due to $H_S+ H_R$ from the slow motion due to $H_{SR}$. Note that we have an assumption that the uncoupled
Hamiltonian is independent of time. Then the form of the evolution of the full density matrix is then given as 
\begin{equation}
    \tilde{\chi}(t) = e^{i\slash \hbar (H_S + H_R)t} \chi (t) e^{-i\slash \hbar (H_S + H_R)t}\label{eq:interaction_picture}
\end{equation}
Using \cref{eq:interaction_picture} and the Liouville von-Neumann equation, we get
\begin{align}
    \dot{\tilde{\chi}}(t) &= \frac{i}{\hbar}\qty(H_S + H_R )\tilde{\chi}(t) - \frac{i}{\hbar} \tilde{\chi}(t) \qty(H_S + H_R) + e^{i\slash \hbar (H_S + H_R)t} \dot{\chi} (t) e^{-i\slash \hbar (H_S + H_R)t} \\ \nonumber
    &= \frac{1}{i\hbar} [\tilde{H_{SR}}(t), \tilde{\chi}(t)] \label{eq:chi_evolution}
\end{align}
Also note that the operator $H_{SR}(t) = e^{i\slash \hbar (H_S + H_R)t} H_{SR} e^{-i\slash \hbar (H_S + H_R)t}$.
Using \cref{eq:chi_evolution}, we can write the evolution of the reduced density matrix as
\begin{align*}
    \tilde{\chi}(t) = \chi (0) + \frac{1}{i \hbar} \int_0^t dt' [\tilde{H_{SR}}(t'), \tilde{\chi}(t')]
\end{align*}
Using this we get,
\begin{align}
    \dot{\tilde{\chi}}(t) = \frac{1}{i \hbar} \qty[\tilde{H}_{SR}(t), \chi (0)] - \frac{1}{\hbar^2} \int_0^t dt' [\tilde{H}_{SR}(t), [\tilde{H}_{SR}(t'), \tilde{\chi}(t')]]
\end{align}
\subsection{Born and Markov Approximation}
To simplify the previous equation to actually understand stuff from it, we use some approximations. So let us set the stage 
for the application of these approximations.

Let the interaction be turned on at $t=0$. The systems are assumed to be initially uncorrelated. This means that the initial density matrix\
$\chi(0) = \tilde{\chi}(0)$ factorizes as 
\begin{equation}
    \chi(0) = \rho(0) R_0
\end{equation}
where $R_0$ is the initial thermal density operator for the reservoir. More on this later. Note that 
\begin{equation*}
    Tr_R[\tilde{\chi}(t)] = e^{i\slash \hbar (H_S + H_R)t} Tr_R[\chi (t)] e^{-i\slash \hbar (H_S + H_R)t} = \tilde{\rho}(t)
\end{equation*}
Tracing over the Hilbert space of the reservoir, 
\begin{equation*}
    \dot{\tilde{\rho}}(t) = -\frac{1}{\hbar^2} \int_0^t dt' Tr_R[\tilde{H}_{SR}(t), [\tilde{H}_{SR}(t'), \tilde{\chi}(t')]]
\end{equation*}
with the assumption of $\frac{1}{i\hbar} Tr_R\qty(\qty[\tilde{H}_{SR}(0),\chi (0)]) = 0$. A note on this 
assumption is provided later.
\begin{definition}[Quantum Master Equation]
    The Quantum Master Equation is given by 
    \begin{equation}
        \dot{\tilde{\rho}}(t) = -\frac{1}{\hbar^2} \int_0^t dt' Tr_R[\tilde{H}_{SR}(t), [\tilde{H}_{SR}(t'), \tilde{\chi}(t')]]
    \end{equation}
    with the assumption of $\frac{1}{i\hbar} Tr_R\qty(\qty[\tilde{H}_{SR}(0),\chi (0)]) = 0$.
\end{definition}
Let us come to the Born approximation. Let us assume that the coupling between the System and the Reservoir is weak. 
This means that the interaction Hamiltonian is small compared to the uncoupled Hamiltonian. Note that the total 
density matrix at time $t=0$ is factorizes as $\chi(0) = \rho(0) R_0$, which means there is no correlation 
between the system and the reservoir at $t=0$. The weak coupling assumption says that $\tilde{\chi}(t)$
shows deviations from the uncorrelated state only upto an order $O(H_{SR})$. Also note that we have assumed that the 
reservoir is quite large. This means that the weak coupling does not appreciably affect the reservoir. 
A note on this relating to reservoir correlation times, is given later.
This weak coupling assumption gives us 
\begin{equation*}
    \tilde{\chi}(t) = \tilde{\rho}(t) R_0  + O(H_{SR})  
\end{equation*}
This is called the Born Approximation.
\begin{definition}[Born Approximation]
    The Born Approximation is the weak coupling assumption which says that the interaction Hamiltonian
    causes the density matrix to deviate from the uncorrelated state only upto an order $O(H_{SR})$.
    Mathematically it is given by 
    \begin{equation}
        \tilde{\chi}(t) = \tilde{\rho}(t) R_0  + O(H_{SR})  
    \end{equation}
\end{definition}
Ignoring terms of order $O(H_{SR}^3)$ in the Quantum Master Equation, we get the Quantum Master Equation in the Born Approximation.
\begin{equation}
    \dot{\tilde{\rho}}(t) = -\frac{1}{\hbar^2} \int_0^t dt' Tr_R[\tilde{H}_{SR}(t), [\tilde{H}_{SR}(t'), \tilde{\rho}(t')R_0]]
\end{equation}
Note that the equation is still non Markovian. The integrand in the above equation, depends on $\tilde{\rho}(t)$ at all times upto $t$.
This non markovian nature is manifested in the fact that the system at a previous time might affect the reservoir, which
can then affect the system at a later time. This causes the system to have a memory of its past, which relates to its current non Markovian nature. 
A note on this will be given after this, which relates to the reservoir correlation times.
\begin{definition}[Markov Approximation]
    The Markov Approximation is the assumption that the system does not have a memory of its past. This means that the system
    does not remember its past states. This is mathematically given by 
    \begin{equation}
        \tilde{\rho}(t') = \tilde{\rho}(t) \quad \forall ~ t' < t
    \end{equation}
    
\end{definition}
\begin{definition}[Born-Markov Master Equation]
    The Born-Markov Master Equation is given by 
    \begin{equation}
        \dot{\tilde{\rho}}(t) = -\frac{1}{\hbar^2} \int_0^t dt' Tr_R[\tilde{H}_{SR}(t), [\tilde{H}_{SR}(t'), \tilde{\rho}(t)R_0]]
    \end{equation}
    with the assumption of $\frac{1}{i\hbar} Tr_R\qty(\qty[\tilde{H}_{SR}(0),\chi (0)]) = 0$. This takes into the account the 
    Born and Markov Approximations.
\end{definition}
The above equation can be expanded to give the Redfield Equation. A derivation is given at \cref{eq:tbd}
\begin{remark}
    Why does $\ihsr (t')$ not cause an issue?
\end{remark}
\subsection{Note on Markov Approximation and Reservoir correlations}
There are physical grounds on which we can justify the Markovian Approximation. The non-Markovian variant manifests in the 
fact that the system can affect the reservoir at some time and the reservoir might affect the system at a later time. This
results in the system being able to remember its past states. Now, here is our assumption of the reservoir being large and having low
correlation times. The reservoir evolves rapidly and cannot remember how it is affected by the system at a past time. Thus, 
the reservoir doesn't allow the retention of memeory of past states. This can happen say when the reservoir
is held at thermal equilibrium. The markov approximation relies on the fact that there are two seperate time scales for the evolution of the composite 
system+reservoir. There is a slow time scale over which the system evolves and a fast time scale over which the reservoir correlation functions decay.

The determination of these time scales are quite involved. A small remark will be included in \cref{eq:tbc} for the determination of the system 
time scale in the case of the Damped Harmonic Oscillator.


\section{The Damped Harmonic Oscillator}
\subsection{The Lindblad Master Equation}
\subsection{Detailed balance Condition}
\subsection{Expectation values for Operators}
\section{Tutorial 1}




\end{document}
