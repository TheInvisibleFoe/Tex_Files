\documentclass[addpoints]{exam}
\usepackage{amsmath,enumitem,wrapfig,amsfonts,eucal}
\usepackage{tikz}
\usepackage{cancel}
\usepackage{physics,mlmodern}
\renewcommand{\thequestionlabel}{Question{question}}
\newcommand{\StudentName}{Sabarno Saha - 22MS037}
\newcommand{\AssignmentName}{Assignment 02}
\pagestyle{headandfoot}
\runningheadrule
\runningheader{MA2202}{\StudentName}{\AssignmentName}
\firstpagefooter{}{}{\thepage}
\runningfooter{}{}{\thepage}
\printanswers
\begin{document}


\par\textbf{IISER Kolkata} \hfill \textbf{Assignment 02}
\vspace{3pt}
\hrule
\vspace{3pt}
\begin{center}
        \LARGE{\textbf{MA2202 :Probability 1}}
\end{center}
\vspace{3pt}

\hrule
\vspace{4pt}
\textbf{Sabarno Saha}, \textbf{22MS037}\hfill \today
\vspace{20pt}
\bigskip
\begin{questions}
    \question \textbf{ Question 1}\\
There are n boxes numbered \(1, 2, \dots , n,\) among which the \(r^{th}\) box contains \(r - 1\) white cubes
and \(n - r\) red cubes. Suppose, we choose a box at random and we remove two cubes from it,
one after another, without replacement.
\begin{enumerate}[label = (\alph*)]
    \item  Find the probability of the second cube being red.
    \item  Find the probability of the second cube being red, given that the first cube is red.
\end{enumerate}
\begin{solution}\\
 
\end{solution}
\newpage
\question \textbf{ Question 2}\\
Let \((\Omega,\mathcal{E},P)\) be a probability space and let \(A_1,A_2,\dots,A_n \in \mathcal{E}\)
and \(P(\bigcap_{i=1}^{n}A_i)\neq 0\). Show that 
\begin{align}
   P(\bigcap_{i=1}^{n}A_i) = P(A_1)P(A_2|A_1)P(A_3|A_1\cap A_2)\dots P(A_n|\bigcap_{i=1}^{n-1}A_i)
\end{align}
\begin{solution}\\
    Note: \(P(\bigcap_{i=1}^{m}A_i) \neq 0\) as \( \bigcap_{i=1}^{n}A_i \subseteq \bigcap_{i=1}^{m}A_i\), 
    thus \(P(\bigcap_{i=1}^{n}A_i\leq P(\bigcap_{i=1}^{m}A_i))\quad \forall m \in {1,2,\dots,n}.\)\\[5]
    \textbf{Method 1:}  
    We solve this by induction. \\[3] 
    \emph{Case n=1} \\ 
    \(P(A_1) = P(A_1)\) which is true.\\
    \emph{Case n = 2}\\ 
    By definition of conditional probability, we have \(P(A_1|A_2) = \dfrac{P(A_1\cap A_2)}{P(A_2)}\).\\
    \underline{Induction Hypothesis for n=k}
    \begin{align}
       P(\bigcap_{i=1}^{k}A_i) = P(A_1)P(A_2|A_1)P(A_3|A_1\cap A_2)\dots P(A_k|\bigcap_{i=1}^{k-1}A_i)
    \end{align}
    We now prove this for n = k+1. Let \(\mathcal{A} = \bigcap_{i=1}^{k}A_i\). By using the definition of
    conditional probability or the case n=2, \(P(A_{k+1}|\mathcal{A}) = \dfrac{P(A_{k+1}\cap \mathcal{A})}{P(\mathcal{A})}\).
    \begin{align*}
        P(\bigcap_{i=1}^{k+1}A_i) &= P(A_{k+1}\cap\mathcal{A})\\ 
                                  &=P(A_{k+1}|\mathcal{A})P(\mathcal{A})\\
                                  &= P(A_{k+1}|\bigcap_{i=1}^{k}A_i)P(\bigcap_{i=1}^{k}A_i)\\
        \tag{Using Induction Hypothesis}
                                  &=P(A_1)P(A_2|A_1)P(A_3|A_1\cap A_2)\dots P(A_k|\bigcap_{i=1}^{k-1}A_i)P(A_{k+1}|\bigcap_{i=1}^{k}A_i)\\ 
                                  &=P(A_1)P(A_2|A_1)P(A_3|A_1\cap A_2)\dots P(A_{k+1}|\bigcap_{i=1}^{k}A_i)
    \end{align*}
    Thus we have proved the equation (1) \(~ \forall~ n\in\mathbb{N}\).\\[5]
    \textbf{Method 2: }
    Let \(B_k = \bigcap_{i=1}^{k}A_i\).
    Then we need to prove that 
    \begin{align*}
        P(B_n) = P(A_1)P(A_2|A_1)P(A_3|B_2)\dots P(A_n|B_{n-1})
    \end{align*}
    We just use our definition of conditional probability i.e. \(P(A|B) = \dfrac{P(A\cap B)}{P(B)}\). 
    \begin{align*}
        P(A_1)P(A_2|A_1)P(A_3|B_2)\dots P(A_n|B_{n-1}) &= P(A_1)\frac{P(A_2\cap A_1)}{P(A_1)} \frac{P(A_3\cap B_2)}{P(B_2)}\dots \frac{P(B_n)}{P(B_{n-1})}\\ 
                                                       &=  \cancel{P(A_1)}\frac{\cancel{P(A_2\cap A_1)}}{\cancel{P(A_1)}} \frac{\cancel{P(A_3\cap B_2)}}{\cancel{P(B_2)}}\dots \frac{P(B_n)}{\cancel{P(B_{n-1})}}\\
                                                       &= P(B_n)\\ 
                                                       &= P(\bigcap_{i=1}^{n}A_i)
    \end{align*}
\end{solution} 

\question \textbf{ Question 3}\\
Let \((\Omega,\mathcal{E},P)\) be a probability space and let \(A_1,A_2,\dots,A_n \in \mathcal{E}\)
be pairwise mutually exclusive. Let \(A = \bigcup_{i=1}^{\infty}A_i\) and \(B \in \mathcal{E}\)
and \(P(B)\neq 0 \). Show that
\begin{align}
    P(A|B) = \sum_{i=1}^{\infty}P(A_i|B)
\end{align}

\begin{solution}\\
 
\end{solution}

\question \textbf{ Question 4}

We are familiar with the famous Monty Hall problem. Now suppose, instead of 3 doors, there
are n doors, only one among which has a prize behind it.
\begin{enumerate}[label=(\alph*)]
  \item Find the probability of winning upon switching given that Monty opens k doors. Will
switching benefit you?
 \item Find the probability of winning upon switching given that Monty opens maximum num-
ber of doors. Will switching benefit you?
    \item  Find the probability of winning upon switching given that Monty opens no doors. Will
switching benefit you?
\end{enumerate}
\begin{solution}\\
 
\end{solution}
\question \textbf{ Question 5}\\
Dropping two points uniformly at random on \([0, 1]\), the unit interval is divided into three segments.
Find the probability that the three segments obtained in this way form a triangle.
\begin{solution}\\
 
\end{solution}
\end{questions}
\end{document}
