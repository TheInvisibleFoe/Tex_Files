%%%%%%%%%%%%%%%%%%%%%%%%%%%%%%%%%%%%%%%%%%%%%%%%%%%%%%%%%%%%%%%
%
% Welcome to Overleaf --- just edit your LaTeX on the left,
% and we'll compile it for you on the right. If you give
% someone the link to this page, they can edit at the same
% time. See the help menu above for more info. Enjoy!
%
%%%%%%%%%%%%%%%%%%%%%%%%%%%%%%%%%%%%%%%%%%%%%%%%%%%%%%%%%%%%%%%
%\title{Math 453 HW 1}
\documentclass[addpoints]{exam}

\usepackage{amsmath,enumitem,wrapfig,amsfonts,eucal}
\usepackage{tikz}
\usepackage{physics,mlmodern}
\renewcommand{\thequestionlabel}{Question{question}}

\newcommand{\StudentName}{Sabarno Saha - 22MS037}
\newcommand{\AssignmentName}{Assignment 02}

\pagestyle{headandfoot}
\runningheadrule
\runningheader{MA2202}{\StudentName}{\AssignmentName}
\firstpagefooter{}{}{\thepage}
\runningfooter{}{}{\thepage}

\printanswers


\begin{document}


\par\textbf{IISER Kolkata} \hfill \textbf{Assignment 02}
\vspace{3pt}
\hrule
\vspace{3pt}
\begin{center}
        \LARGE{\textbf{MA2202 :Probability 1}}
\end{center}
\vspace{3pt}

\hrule
\vspace{4pt}
\textbf{Sabarno Saha}, \textbf{22MS037}\hfill \today

\vspace{20pt}

\bigskip

\begin{questions}
    \questionlabel \textbf{ Question 1}\\

There are n boxes numbered \(1, 2, \dots , n,\) among which the \(r^{th}\) box contains \(r - 1\) white cubes
and \(n - r\) red cubes. Suppose, we choose a box at random and we remove two cubes from it,
one after another, without replacement.
\begin{enumerate}[label = (\alph*)]
    \item  Find the probability of the second cube being red.
    \item  Find the probability of the second cube being red, given that the first cube is red.
\end{enumerate}


\begin{solution}\\
 
\end{solution}

\question \textbf{ Question 2}\\
Let \((\Omega,\mathcal{E},P)\) be a probability space and let \(A_1,A_2,\dots,A_n \in \mathcal{E}\)
and \(P(\bigcap_{i=1}^{n}A_i)\neq 0\). Show that 
\begin{align*}
   P(\bigcap_{i=1}^{n}A_i) = P(A_1)P(A_2|A_1)P(A_3|A_1\cap A_2)\dots P(A_n|\bigcap_{i=1}^{n-1}A_i)
\end{align*}

\begin{solution}\\
 
\end{solution}

\question \textbf{ Question 3}\\
Let \((\Omega,\mathcal{E},P)\) be a probability space and let \(A_1,A_2,\dots,A_n \in \mathcal{E}\)
be pairwise mutually exclusive. Let \(A = \bigcup_{i=1}^{\infty}A_i\) and \(B \in \mathcal{E}\)
and \(P(B)\neq 0 \). Show that
\begin{align}
    P(A|B) = \sum_{i=1}^{\infty}P(A_i|B)
\end{align}

\begin{solution}\\
 
\end{solution}

\question \textbf{ Question 4}

We are familiar with the famous Monty Hall problem. Now suppose, instead of 3 doors, there
are n doors, only one among which has a prize behind it.
\begin{enumerate}[label=(\alph*)]
  \item Find the probability of winning upon switching given that Monty opens k doors. Will
switching benefit you?
 \item Find the probability of winning upon switching given that Monty opens maximum num-
ber of doors. Will switching benefit you?
    \item  Find the probability of winning upon switching given that Monty opens no doors. Will
switching benefit you?
\end{enumerate}
\begin{solution}\\
 
\end{solution}
\question \textbf{ Question 5}\\
Dropping two points uniformly at random on \([0, 1]\), the unit interval is divided into three segments.
Find the probability that the three segments obtained in this way form a triangle.
\begin{solution}\\
 
\end{solution}
\end{questions}
\end{document}
