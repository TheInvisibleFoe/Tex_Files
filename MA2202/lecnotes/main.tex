\documentclass[a4paper]{article}

\usepackage[utf8]{inputenc}
\usepackage[T1]{fontenc}
\usepackage{mlmodern}
\usepackage{textcomp}
\usepackage[dutch]{babel}
\usepackage{amsmath, amssymb}
\usepackage{preamble}
\usepackage{transparent}
\newcommand{\incfig}[1]{%
    \def\svgwidth{\columnwidth}
    \import{./figures/}{#1.pdf_tex}
}
\pdfsuppresswarningpagegroup=1
\title{Probability 1}
\begin{document}
% \maketitle
\begin{titlepage}

\newcommand{\HRule}{\rule{\linewidth}{0.5mm}} % Defines a new command for the horizontal lines, change thickness here

\center % Center everything on the page
 
%----------------------------------------------------------------------------------------
%	HEADING SECTIONS
%----------------------------------------------------------------------------------------

\textsc{\LARGE IISERK}\\[1.5cm] % Name of your university/college
% \textsc{\Large  }\\[0.5cm] % Major heading such as course name
% \textsc{\large Minor Heading}\\[0.5cm] % Minor heading such as course title

%----------------------------------------------------------------------------------------
%	TITLE SECTION
%----------------------------------------------------------------------------------------

\HRule \\[0.4cm]
{ \huge \bfseries Probability 1}\\[0.4cm] % Title of your document
\HRule \\[1.5cm]
 
%----------------------------------------------------------------------------------------
%	AUTHOR SECTION
%----------------------------------------------------------------------------------------

\begin{minipage}{0.4\textwidth}
\begin{flushleft} \large
\emph{Author:}\\
Sabarno \textsc{Saha} % Your name
\end{flushleft}
\end{minipage}
~
\begin{minipage}{0.4\textwidth}
\begin{flushright} \large
\emph{Instructor:} \\
Dr. Soumya \textsc{Bhattacharya} % Supervisor's Name
\end{flushright}
\end{minipage}\\[2cm]

% If you don't want a supervisor, uncomment the two lines below and remove the section above
%\Large \emph{Author:}\\
%John \textsc{Smith}\\[3cm] % Your name

%----------------------------------------------------------------------------------------
%	DATE SECTION
%----------------------------------------------------------------------------------------

{\large \today}\\[2cm] % Date, change the \today to a set date if you want to be precise

%----------------------------------------------------------------------------------------
%	LOGO SECTION
%----------------------------------------------------------------------------------------

% \includegraphics{logo.png}\\[1cm] % Include a department/university logo - this will require the graphicx package
 
%----------------------------------------------------------------------------------------

\vfill % Fill the rest of the page with whitespace

\end{titlepage}



\tableofcontents
\newpage

\section{Introduction}
This is a first course in Probability covering the following topics.
\begin{itemize}
  \item Basic Axioms of Probability
  \item Random Variables
  \item Inequalities
\end{itemize}
The reference we will be using is \emph{A First Course in Probability} by Sheldon Ross.
\section{The Birthday problem}
\begin{problem}[The Birthday Problem]
    Given n randomly chosen people in a room, what is the probability that two people share a birthday? A similar 
    restatement can be given that the probability that people share a birthday is above 50\%, 
    what will be the value of n (i.e. no of people in the room)?
\end{problem}
A somewhat counterintuitive answer for the restatement is that we only need 23 randomly chosen 
people in a room to get the probability of a birthday match to be above 50\%. We will actually 
plot a graph and show the probability for n people.
\begin{explanation}
  The proof of the answer is quite simple. Let E be the event that two people in the n people 
  share a birthday. Then we can easily calculate \(E^c\).
 \begin{align*}
     P(E^c) & = \frac{365}{365} \cdot \frac{364}{365} \cdots \frac{365-n}{365} \\ 
            & = \frac{^{365}P_n}{365^{23}} \\ 
     P(E) &= 1 - P(E^c)\\ 
          &= 1- \frac{^{365}P_n}{365^{23}}
 \end{align*}
 If we  put n = 23  we get P(\(E^c) \approx\)  0.507 which proves our result.
\end{explanation}
A graph \(P(E^c)\) vs n is shown below.
We can extend to something called the Probabilistic Pigeonhole Principle.
\begin{theorem}[The Probabilistic Pigeonhole Principle]
  Given n balls and m bins, and the probability that atleast two balls are in the same bin is 
  given by the inequality:
 \begin{align}
     n > \frac{1}{2} + \sqrt{2m\log(\frac{1}{1-p})+\frac{1}{4}}
 \end{align}
\end{theorem}
\begin{explanation}
    We use a fairly random inequality \(e^{-x}\ge1-x \quad \forall x\in \R \).
\end{explanation}
\section{Axioms of Probability}
\subsection{Definitions}
\begin{itemize}
    \item \emph{Random Experiment: } A random experiment is an experiment where a certain outcome
        has no effect on other outcomes when repeated multiple times.
    \item \emph{Mutually Exclusive: }
    \item \emph{Exhaustive: }
    \item \emph{Sample Space \(\Omega\): } A sample space is the set of all outcomes of a Random Experiment.
    \item \emph{Event Space \(\epsilon\): }
\end{itemize}
\subsection{Axioms}


\begin{itemize}
    \item If \(A\in\epsilon\),then \(A^c\in\epsilon\).
    \item \(\epsilon\) is closed under countable union.
    \item Let $P:\epsilon\rightarrow[0,1]$ be a function st if A and B are events s.t. 
       \(A,B\in \epsilon \) where \(A\cap B = \phi\) then 
       \(P(A\cup B) = P(A)+P(B)\).\\ 
       In general, if \(A_1, \dots, A_n\) are mutually exclusive, then \\ 
       \begin{center}
           \(P(\bigcup\limits_{i=1}^{n}A_i) = \sum\limits_{i=1}^{n}P(A_i)\)
       \end{center}
\end{itemize}
\section{Run of heads}
\section{Boole's Inequality}
\section{Bon-Ferroni's Inequality}
\section{Conditional Probability}

\section{Monty Hall}




\end{document}
